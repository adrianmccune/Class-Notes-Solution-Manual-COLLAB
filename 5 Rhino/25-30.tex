%%%%%%%%%%%%%%%%%%%%%%%%%%%%%%%%%%%%%%
%
\fbox{\begin{minipage}[t]{.465 \textwidth}
\index{({ 5.0\theproblem} )\; Additive property of equality with signed fractions}
% type problem below this line
Solve for $w$. 
$$
-3 = w - \frac{3}{8}
$$

Simplify your answer as much as possible.



%type problem above this line
\ifbool{answerKey}%
{\textbf{Explanation:}
\begin {itemize}
\item The variable $w$ has $\frac{3}{8}$ subtracted from it, so we undo that by adding $\frac{3}{8}$ to both sides.
$$ -3 + \frac{3}{8} = w - \frac{3}{8} + \frac{3}{8} $$
\item Rewrite the expression.
$$ w = -3 + \frac{3}{8}$$
\item Turn $-3$ into a fraction with the same denominator as $\frac{3}{8}$ by multiplying with the appropriate form of one.
$$ w = - \frac {3}{1} \times \frac {8}{8} + \frac{3}{8}$$
$$ w = - \frac {24}{8} + \frac{3}{8}$$
\item Add the numerators to combine into a single fraction.
\end {itemize}}
\vfill \textbf{Answer:}
%type answer below this line
$w = -\frac{21}{8}$



%type answer above this line
}%
{}
\vfill\hfill
{\scriptsize 5.0\theproblem}
\stepcounter{problem}
\end{minipage}}%
%
%%%%%%%%%%%%%%%%%%%%%%%%%%%%%%%%%%%%%%
%
\fbox{\begin{minipage}[t]{.465 \textwidth}
\index{({ 5.0\theproblem} )\; Ordering real numbers}
% type problem below this line

Order these numbers from least to greatest.
$$ \frac{133}{10},\; -13.\overline{4},\; -13\frac{5}{11},\; -\sqrt{180} $$





%type problem above this line
\ifbool{answerKey}%
{\textbf{Explanation:}
\begin {itemize}
\item Convert all numbers to decimals using calculator.
$$ frac {133}{10} = 13.3, -13.\overline{4}, -13\frac{5}{11} = -13.\overline{45}, -\sqrt{180} \approx {-13.416} $$
\item Order the numbers least to greatest.
$$ -13.\overline{45}, -13.\overline{4}, {-13.416}, 13.3 $$
\item Rewrite original form of numbers.
\end {itemize}}
\vfill \textbf{Answer:}
%type answer below this line

$$-13\frac{5}{11} < -13.\overline{4} < -\sqrt{180} < \frac{133}{10}$$



%type answer above this line
}%
{}
\vfill\hfill
{\scriptsize 5.0\theproblem}
\stepcounter{problem}
\end{minipage}}

%
%%%%%%%%%%%%%%%%%%%%%%%%%%%%%%%%%%%%%%
%

\fbox{\begin{minipage}[t]{.465 \textwidth}
\index{({ 5.0\theproblem} )\; Integer multiplication and division}
% type problem below this line
Evaluate the following.
\vspace{.2 cm}

(a) $-5 \times 3 =$
\vspace{.2 cm}

(b) $42 \div (-6) =$





%type problem above this line
\ifbool{answerKey}%
{\textbf{Explanation:}
\begin {itemize}
\item a) Multiply the numbers, leaving negative as is because $- \times +$ is $-$.
\item b) Divide the number $42$ by $6$, leaving negative sign as is. $+ \div -$ is $-$. Rewrite final answer with negative sign.
\end {itemize}}
\vfill \textbf{Answer:}
%type answer below this line

(a) $-15$

(b) $-7$



%type answer above this line
}%
{}
\vfill\hfill
{\scriptsize 5.0\theproblem}
\stepcounter{problem}
\end{minipage}}%
%
%%%%%%%%%%%%%%%%%%%%%%%%%%%%%%%%%%%%%%
%
\fbox{\begin{minipage}[t]{.465 \textwidth}
\index{({ 5.0\theproblem} )\; Multiplication of 3 or 4 integers}
% type problem below this line
Evaluate.

$$-1(-4)(3)(3)$$




%type problem above this line
\ifbool{answerKey}%
{\textbf{Explanation:}
\begin {itemize}
\item Multiply the negative numbers together ($- \times +$ is $-$).
$$-1 \times -4 = 4$$
\item Multiply the remaining numbers out.
$$4 \times 3 \times 3$$
\end {itemize}}
\vfill \textbf{Answer:}
%type answer below this line
$36$



%type answer above this line
}%
{}
\vfill\hfill
{\scriptsize 5.0\theproblem}
\stepcounter{problem}
\end{minipage}}

%
%%%%%%%%%%%%%%%%%%%%%%%%%%%%%%%%%%%%%%
%

\fbox{\begin{minipage}[t]{.465 \textwidth}
\index{({ 5.0\theproblem} )\; Identifying numbers as integers or non-integers}
% type problem below this line
Classify each number below as an integer or not.

$$-\frac{17}{4},\; -\frac{63}{9},\; 92,\; -35.67,\; -417.32$$




%type problem above this line
\ifbool{answerKey}%
{\textbf{Explanation:}
\begin {itemize}
\item Classify the numbers knowing that an integer is a whole number and does not include fractions or decimals.
\end {itemize}}
\vfill \textbf{Answer:}
%type answer below this line

Integers: $-\frac{63}{9},\; 92$

Not: $-\frac{17}{4},\; -35.67,\; -417.32$




%type answer above this line
}%
{}
\vfill\hfill
{\scriptsize 5.0\theproblem}
\stepcounter{problem}
\end{minipage}}%
%
%%%%%%%%%%%%%%%%%%%%%%%%%%%%%%%%%%%%%%
%
\fbox{\begin{minipage}[t]{.465 \textwidth}
\index{({ 5.0\theproblem} )\; Identifying numbers as rational or irrational}
% type problem below this line
Classify each number below as a rational number or an irrational number.

$$-\pi,\; -51.\overline{85},\; \sqrt{13},\; -\frac{13}{14},\; \sqrt{1}$$





%type problem above this line
\ifbool{answerKey}%
{\textbf{Explanation:}
\begin {itemize}
A rational number is any number that can be written as a fraction, or any decimal that ends or repeats.
\item $-\pi$ is a number with decimals that do not end and never repeat so it is irrational.
\item $-51.\overline{85}$ is a repeating decimal so it is rational.
\item $\sqrt{13}$ When converted to a decimal, it does not end and does not repeat so it is irrational.
\item $-\frac{13}{14}$ is a fraction, so it is rational.
\item $\sqrt{1} = 1$, which can be written as $\frac{1}{1}$ so it is rational. 
\end {itemize}}
\vfill \textbf{Answer:}
%type answer below this line

Rational: $-51.\overline{85},\; -\frac{13}{14},\; \sqrt{1}$

Irrational: $-\pi,\; \sqrt{13}$



%type answer above this line
}%
{}
\vfill\hfill
{\scriptsize 5.0\theproblem}
\stepcounter{problem}
\end{minipage}}
