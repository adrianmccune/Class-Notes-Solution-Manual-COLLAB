%%%%%%%%%%%%%%%%%%%%%%%%%%%%%%%%%%%%%%
%
\fbox{\begin{minipage}[t]{.465 \textwidth}
\index{({ 5.0\theproblem} )\; Square root of a perfect square}
% type problem below this line

What is the value of $\sqrt{49}$?




%type problem above this line
\ifbool{answerKey}%
{\textbf{Explanation:}
\begin {itemize}
\item Rewrite the number inside the square root as an exponential number.
$$\sqrt {7 \times 7}$$
\item Taking the square root of a perfect square number is the number itself.
\end {itemize}}
\vfill \textbf{Answer:}
%type answer below this line
$7$



%type answer above this line
}%
{}
\vfill\hfill
{\scriptsize 5.0\theproblem}
\stepcounter{problem}
\end{minipage}}%
%
%%%%%%%%%%%%%%%%%%%%%%%%%%%%%%%%%%%%%%
%
\fbox{\begin{minipage}[t]{.465 \textwidth}
\index{({ 5.0\theproblem} )\; Using a calculator to approximate a square root}
% type problem below this line
Use a calculator to approximate $\sqrt{13.6}$

Round your answer to the nearest hundredth.





%type problem above this line
\ifbool{answerKey}%
{\textbf{Explanation:}
\begin {itemize}
\item Enter $\sqrt{13.6}$ into calculator and evaluate.
\item Round to second decimal place. 
\end {itemize}}
\vfill \textbf{Answer:}
%type answer below this line
$3.69$



%type answer above this line
}%
{}
\vfill\hfill
{\scriptsize 5.0\theproblem}
\stepcounter{problem}
\end{minipage}}

%
%%%%%%%%%%%%%%%%%%%%%%%%%%%%%%%%%%%%%%
%

\fbox{\begin{minipage}[t]{.465 \textwidth}
\index{({ 5.0\theproblem} )\; Estimating a square root}
% type problem below this line

Find two consecutive whole numbers that $\sqrt{18}$ lies between.




%type problem above this line
\ifbool{answerKey}%
{\textbf{Explanation:}
\begin {itemize} 
\item Find the perfect squares 18 is in between.
$$4^2=16 and 5^2=25$$
\item When the square root is taken, it will be a decimal number between $4$ and $5$.
\end {itemize}}
\vfill \textbf{Answer:}
%type answer below this line
$4$ and $5$



%type answer above this line
}%
{}
\vfill\hfill
{\scriptsize 5.0\theproblem}
\stepcounter{problem}
\end{minipage}}%
%
%%%%%%%%%%%%%%%%%%%%%%%%%%%%%%%%%%%%%%
%
\fbox{\begin{minipage}[t]{.465 \textwidth}
\index{({ 5.0\theproblem} )\; Introduction to the Pythagorean Theorem}
% type problem below this line

For the following right triangle, find the side length $x$.

\begin{center}
    \includegraphics[width=.55\linewidth]{images/Rhino16.PNG}
\end{center}



%type problem above this line
\ifbool{answerKey}%
{\textbf{Explanation:}
\begin {itemize}
\item Identify the hypotenuse, which in this case is the side marked $x$.
\item Use the Pythagorean Theorem to set up an equation relating the three sides of the triangle, with c being the hypotenuse.
$$ a^2 + b^2 = c^2 
35^2 + 12^2 = x^2 $$
\item Solve the left side.
$$1225 + 144 = x^2 
x^2 = 1369$$
\item To solve for $x$, take square root on both sides.
\sqrt {x^2} = \sqrt {1369}
\end {itemize}}
\vfill \textbf{Answer:}
%type answer below this line
$37$



%type answer above this line
}%
{}
\vfill\hfill
{\scriptsize 5.0\theproblem}
\stepcounter{problem}
\end{minipage}}

%
%%%%%%%%%%%%%%%%%%%%%%%%%%%%%%%%%%%%%%
%

\fbox{\begin{minipage}[t][2.85 in]{.465 \textwidth}
\index{({ 5.0\theproblem} )\; Pythagorean Theorem}
% type problem below this line

For the following right triangle, find the side length $x$. Round your answer to the nearest hundredth.

\begin{center}
    \includegraphics[width=.5\linewidth]{images/Rhino17.PNG}
\end{center}




%type problem above this line
\ifbool{answerKey}%
{\textbf{Explanation:}
\begin {itemize}
\item Set up an equation with the Pythagorean Theorem relating the three sides.
$$ a^2 + b^2 = c^2 
9^2 + x^2 = 17^2 $$
\item Isolate $x^2$
$$x^2 = 17^2 - 9^2 
x^2 = 208$$
\item Take square root on both sides to solve for x.
\sqrt {x^2} = \sqrt {208}
\end {itemize}}
\item Round to 2 decimal places.
\vfill \textbf{Answer:}
%type answer below this line
$14.42$



%type answer above this line
}%
{}
\vfill\hfill
{\scriptsize 5.0\theproblem}
\stepcounter{problem}
\end{minipage}}%
%
%%%%%%%%%%%%%%%%%%%%%%%%%%%%%%%%%%%%%%
%
\fbox{\begin{minipage}[t]{.465 \textwidth}
\index{({ 5.0\theproblem} )\; Multiplicative property of equality with decimals
}
% type problem below this line
Solve for $u$. 
$$
    3.75 = 3u
$$




%type problem above this line
\ifbool{answerKey}%
{\textbf{Explanation:}
\begin {itemize}
\item The variable $u$ is multiplied by $3$, so we undo that by dividing both sides by $3$.
$$ \frac {3.75}{3} = \frac {3u}{3} $$
\item Cancel $3$ on right side due to equivalence.
$$u = \frac {3.75}{3}$$
\item Divide $3.75$ by $3$.
\end {itemize}}
\vfill \textbf{Answer:}
%type answer below this line
$u = 1.25$




%type answer above this line
}%
{}
\vfill\hfill
{\scriptsize 5.0\theproblem}
\stepcounter{problem}
\end{minipage}}
