\vspace{1 cm}

\textbf{The following topics are not included in the Class Notes but can be found in ALEKS}

\vspace{.2cm}

The following ALEKS topics are also included in Math 100, but sample problems are not included in the class notes. If you have would like help finding these topics to practice, come to office hours and ask a Math 100 TA.

\begin{itemize}
    \item ({ 6.\theproblem} )\; Computing unit prices to find the better buy
    \index{({ 6.\theproblem} )\; Computing unit prices to find the better buy}
    \stepcounter{problem}

    \item ({ 6.\theproblem} )\; Properties of addition
    \index{({ 6.\theproblem} )\; Properties of addition}
    \stepcounter{problem}

    \item ({ 6.\theproblem} )\; Properties of real numbers
    \index{({ 6.\theproblem} )\; Properties of real numbers}
    \stepcounter{problem}

    \item ({ 6.\theproblem} )\; Identifying properties used to solve a linear equation
    \index{({ 6.\theproblem} )\; Identifying properties used to solve a linear equation}
    \stepcounter{problem}

    \item ({ 6.\theproblem} )\; Solving equations with zero, one, or infinitely many solutions
    \index{({ 6.\theproblem} )\; Solving equations with zero, one, or infinitely many solutions}
    \stepcounter{problem}

    \item ({ 6.\theproblem} )\; Graphing a linear equation of the form $y = mx$
    \index{({ 6.\theproblem} )\; Graphing a linear equation of the form $y = mx$}
    \stepcounter{problem}

    \item ({ 6.\theproblem} )\; Graphing a line given its equation in slope-intercept form: Integer slope
    \index{({ 6.\theproblem} )\; Graphing a line given its equation in slope-intercept form: Integer slope}
    \stepcounter{problem}

    \item ({ 6.\theproblem} )\; Graphing a line given its equation in standard form
    \index{({ 6.\theproblem} )\; Graphing a line given its equation in standard form}
    \stepcounter{problem}

    \item ({ 6.\theproblem} )\; Graphing a line by first finding its x- and y-intercepts
    \index{({ 6.\theproblem} )\; Graphing a line by first finding its x- and y-intercepts}
    \stepcounter{problem}

    \item ({ 6.\theproblem} )\; Classifying slopes given graphs of lines
    \index{({ 6.\theproblem} )\; Classifying slopes given graphs of lines}
    \stepcounter{problem}

    \item ({ 6.\theproblem} )\; Graphing a line given its slope and y-intercept
    \index{({ 6.\theproblem} )\; Graphing a line given its slope and y-intercept}
    \stepcounter{problem}

    \item ({ 6.\theproblem} )\; Graphing a line through a given point with a given slope
    \index{({ 6.\theproblem} )\; Graphing a line through a given point with a given slope}
    \stepcounter{problem}

    \item ({ 6.\theproblem} )\; Identifying linear functions given ordered pairs
    \index{({ 6.\theproblem} )\; Identifying linear functions given ordered pairs}
    \stepcounter{problem}

    \item ({ 6.\theproblem} )\; Graphing a line by first finding its slope and y-intercept
    \index{({ 6.\theproblem} )\; Graphing a line by first finding its slope and y-intercept}
    \stepcounter{problem}

    \item ({ 6.\theproblem} )\; Writing an equation and graphing a line given its slope and y-intercept
    \index{({ 6.\theproblem} )\; Writing an equation and graphing a line given its slope and y-intercept}
    \stepcounter{problem}

    \item ({ 6.\theproblem} )\; Writing an equation and drawing its graph to model a real-world situation: Advanced
    \index{({ 6.\theproblem} )\; Writing an equation and drawing its graph to model a real-world situation: Advanced}
    \stepcounter{problem}

    \item ({ 6.\theproblem} )\; Combining functions to write a new function that models a real-world situation
    \index{({ 6.\theproblem} )\; Combining functions to write a new function that models a real-world situation}
    \stepcounter{problem}

    \item ({ 6.\theproblem} )\; Comparing properties of linear functions given in different forms
    \index{({ 6.\theproblem} )\; Comparing properties of linear functions given in different forms}
    \stepcounter{problem}

    \item ({ 6.\theproblem} )\; Interpreting the parameters of a linear function that models a real-world situation
    \index{({ 6.\theproblem} )\; Interpreting the parameters of a linear function that models a real-world situation}
    \stepcounter{problem}

    \item ({ 6.\theproblem} )\; Application problem with a linear function: Finding a coordinate given the slope and a point
    \index{({ 6.\theproblem} )\; Application problem with a linear function: Finding a coordinate given the slope and a point}
    \stepcounter{problem}

    \item ({ 6.\theproblem} )\; Application problem with a linear function: Finding a coordinate given two points
    \index{({ 6.\theproblem} )\; Application problem with a linear function: Finding a coordinate given two points}
    \stepcounter{problem}

    \item ({ 6.\theproblem} )\; Identifying independent and dependent variables from equations or real-world situations
    \index{({ 6.\theproblem} )\; Identifying independent and dependent variables from equations or real-world situations}
    \stepcounter{problem}

    \item ({ 6.\theproblem} )\; Solving a linear equation by graphing
    \index{({ 6.\theproblem} )\; Solving a linear equation by graphing}
    \stepcounter{problem}

    \item ({ 6.\theproblem} )\; Identifying functions from relations
    \index{({ 6.\theproblem} )\; Identifying functions from relations}
    \stepcounter{problem}

    \item ({ 6.\theproblem} )\; Vertical line test
    \index{({ 6.\theproblem} )\; Vertical line test}
    \stepcounter{problem}

    \item ({ 6.\theproblem} )\; Table for a linear function
    \index{({ 6.\theproblem} )\; Table for a linear function}
    \stepcounter{problem}

    \item ({ 6.\theproblem} )\; Evaluating a piecewise-defined function
    \index{({ 6.\theproblem} )\; Evaluating a piecewise-defined function}
    \stepcounter{problem}

    \item ({ 6.\theproblem} )\; Determining whether an equation defines a function: Basic
    \index{({ 6.\theproblem} )\; Determining whether an equation defines a function: Basic}
    \stepcounter{problem}

    \item ({ 6.\theproblem} )\; Domain and range of a linear function that models a real-world situation
    \index{({ 6.\theproblem} )\; Domain and range of a linear function that models a real-world situation}
    \stepcounter{problem}

    \item ({ 6.\theproblem} )\; Domain and range from the graph of a continuous function
    \index{({ 6.\theproblem} )\; Domain and range from the graph of a continuous function}
    \stepcounter{problem}

    \item ({ 6.\theproblem} )\; Finding an output of a function from its graph
    \index{({ 6.\theproblem} )\; Finding an output of a function from its graph}
    \stepcounter{problem}

    \item ({ 6.\theproblem} )\; Finding inputs and outputs of a function from its graph
    \index{({ 6.\theproblem} )\; Finding inputs and outputs of a function from its graph}
    \stepcounter{problem}

    \item ({ 6.\theproblem} )\; Finding where a function is increasing, decreasing, or constant given the graph
    \index{({ 6.\theproblem} )\; Finding where a function is increasing, decreasing, or constant given the graph}
    \stepcounter{problem}

    \item ({ 6.\theproblem} )\; Choosing a graph to fit a narrative: Basic
    \index{({ 6.\theproblem} )\; Choosing a graph to fit a narrative: Basic}
    \stepcounter{problem}

    \item ({ 6.\theproblem} )\; Graphing an integer function and finding its range for a given domain
    \index{({ 6.\theproblem} )\; Graphing an integer function and finding its range for a given domain}
    \stepcounter{problem}

    \item ({ 6.\theproblem} )\; Graphing a function of the form $f(x) = ax + b$: Integer slope
    \index{({ 6.\theproblem} )\; Graphing a function of the form $f(x) = ax + b$: Integer slope}
    \stepcounter{problem}

    \item ({ 6.\theproblem} )\; Graphing a function of the form $f(x) = ax + b$: Fractional slope
    \index{({ 6.\theproblem} )\; Graphing a function of the form $f(x) = ax + b$: Fractional slope}
    \stepcounter{problem}

    \item ({ 6.\theproblem} )\; Graphically solving a system of linear equations
    \index{({ 6.\theproblem} )\; Graphically solving a system of linear equations}
    \stepcounter{problem}

    \item ({ 6.\theproblem} )\; Interpreting the graphs of two functions
    \index{({ 6.\theproblem} )\; Interpreting the graphs of two functions}
    \stepcounter{problem}

    
\end{itemize}

