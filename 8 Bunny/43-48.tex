%%%%%%%%%%%%%%%%%%%%%%%%%%%%%%%%%%%%%%
%
\fbox{\begin{minipage}[t][2.85 in]{.465 \textwidth}
\index{({ 8.0\theproblem} )\; Rational exponents: Quotient rule}
% 8.43. type problem below this line

Simplify. Write your answer using only a positive exponent.
Assume that the variable represents a positive real number.

$$\frac{w^\frac{1}{2}}{w^\frac{5}{8}}$$



%type problem above this line
\ifbool{answerKey}%
{
\vfill \textbf{Answer:}
%type answer below this line

$$\frac{1}{w^\frac{1}{8}}$$



%type answer above this line
}%
{}
\vfill\hfill
{\scriptsize 8.0\theproblem}
\stepcounter{problem}
\end{minipage}}%
%
%%%%%%%%%%%%%%%%%%%%%%%%%%%%%%%%%%%%%%
%
\fbox{\begin{minipage}[t][2.85 in]{.465 \textwidth}
\index{({ 8.0\theproblem} )\; Rational exponents: Products and quotients with negative exponents}
% 8.44. type problem below this line

Simplify the expression. Write your answer using only positive exponents.
Assume that all variables are positive real numbers.

$$\frac{\phantom{A} x^\frac{-1}{2} x^\frac{5}{2} \phantom{A}}{x^\frac{1}{3}}$$



%type problem above this line
\ifbool{answerKey}%
{
\vfill \textbf{Answer:}
%type answer below this line

$$x^\frac{5}{3}$$



%type answer above this line
}%
{}
\vfill\hfill
{\scriptsize 8.0\theproblem}
\stepcounter{problem}
\end{minipage}}

%
%%%%%%%%%%%%%%%%%%%%%%%%%%%%%%%%%%%%%%
%

\fbox{\begin{minipage}[t][2.85 in]{.465 \textwidth}
\index{({ 8.0\theproblem} )\; Rational exponents: Power of a power rule}
% 8.45. type problem below this line

Simplify.
$$\left( y^\frac{8}{15} \right)^3$$
Write your answer without parentheses.
Assume that the variable represents a positive real number.


%type problem above this line
\ifbool{answerKey}%
{
\vfill \textbf{Answer:}
%type answer below this line

$$y^\frac{8}{5}$$



%type answer above this line
}%
{}
\vfill\hfill
{\scriptsize 8.0\theproblem}
\stepcounter{problem}
\end{minipage}}%
%
%%%%%%%%%%%%%%%%%%%%%%%%%%%%%%%%%%%%%%
%
\fbox{\begin{minipage}[t][2.85 in]{.465 \textwidth}
\index{({ 8.0\theproblem} )\; Rational exponents: Powers of powers with negative exponents}
% 8.46. type problem below this line

Simplify the expression.
$$\left( a^\frac{1}{5} \times c^\frac{-2}{3} \right)^4$$
Write your answer without using negative exponents.
Assume that all variables are positive real numbers.


%type problem above this line
\ifbool{answerKey}%
{
\vfill \textbf{Answer:}
%type answer below this line

$$ \frac{a^\frac{4}{5}}{c^\frac{8}{3}}$$



%type answer above this line
}%
{}
\vfill\hfill
{\scriptsize 8.0\theproblem}
\stepcounter{problem}
\end{minipage}}

%
%%%%%%%%%%%%%%%%%%%%%%%%%%%%%%%%%%%%%%
%

\fbox{\begin{minipage}[t][2.85 in]{.465 \textwidth}
\index{({ 8.0\theproblem} )\; Introduction to square root addition or subtraction}
% 8.47. type problem below this line

Simplify.
$$\sqrt{5} + 4\sqrt{5}$$



%type problem above this line
\ifbool{answerKey}%
{
\vfill \textbf{Answer:}
%type answer below this line
$5\sqrt{5}$



%type answer above this line
}%
{}
\vfill\hfill
{\scriptsize 8.0\theproblem}
\stepcounter{problem}
\end{minipage}}%
%
%%%%%%%%%%%%%%%%%%%%%%%%%%%%%%%%%%%%%%
%
\fbox{\begin{minipage}[t][2.85 in]{.465 \textwidth}
\index{({ 8.0\theproblem} )\; Introduction to square root multiplication}
% 8.48. type problem below this line

Simplify.
$$\sqrt{5} \times \sqrt{6}$$



%type problem above this line
\ifbool{answerKey}%
{
\vfill \textbf{Answer:}
%type answer below this line
$\sqrt{30}$



%type answer above this line
}%
{}
\vfill\hfill
{\scriptsize 8.0\theproblem}
\stepcounter{problem}
\end{minipage}}