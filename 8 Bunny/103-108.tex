%%%%%%%%%%%%%%%%%%%%%%%%%%%%%%%%%%%%%%
%
\fbox{\begin{minipage}[t][2.85 in]{.465 \textwidth}
\index{({ 8.0\theproblem} )\; Expanding a logarithmic expression: Problem type 2}
% 103. type problem below this line

Use the properties of logarithms to expand.

Each logarithm should involve only one variable and should not have any radicals or exponents.

You may assume that all variables are positive.

$$\log{\sqrt{x^7yz^3}}$$


%type problem above this line
\ifbool{answerKey}%
{
\vfill \textbf{Answer:}
%type answer below this line

$$\frac{7}{2}\log{x} + \frac{1}{2}\log{y} + \frac{3}{2}\log{z}$$



%type answer above this line
}%
{}
\vfill\hfill
{\scriptsize 8.0\theproblem}
\stepcounter{problem}
\end{minipage}}%
%
%%%%%%%%%%%%%%%%%%%%%%%%%%%%%%%%%%%%%%
%
\fbox{\begin{minipage}[t][2.85 in]{.465 \textwidth}
\index{({ 8.0\theproblem} )\; Writing an expression as a single logarithm}
% 104. type problem below this line

Write the expression as a single logarithm.

$$3\log_a{(x-4)} -5\log_a{(x+7)}$$


%type problem above this line
\ifbool{answerKey}%
{
\vfill \textbf{Answer:}
%type answer below this line

$$\log_a{\left( \frac{(x-4)^3}{(x+7)^5} \right)}$$



%type answer above this line
}%
{}
\vfill\hfill
{\scriptsize 8.0\theproblem}
\stepcounter{problem}
\end{minipage}}

%
%%%%%%%%%%%%%%%%%%%%%%%%%%%%%%%%%%%%%%
%

\fbox{\begin{minipage}[t][2.85 in]{.465 \textwidth}
\index{({ 8.0\theproblem} )\; Change of base for logarithms: Problem type 1}
% 105. type problem below this line

Use the change of base formula to compute $$log_{\frac{1}{7}}{\frac{1}{5}}$$

Round your answer to the nearest thousandth.



%type problem above this line
\ifbool{answerKey}%
{
\vfill \textbf{Answer:}
%type answer below this line
$0.8273$



%type answer above this line
}%
{}
\vfill\hfill
{\scriptsize 8.0\theproblem}
\stepcounter{problem}
\end{minipage}}%
%
%%%%%%%%%%%%%%%%%%%%%%%%%%%%%%%%%%%%%%
%
\fbox{\begin{minipage}[t][2.85 in]{.465 \textwidth}
\index{({ 8.0\theproblem} )\; Change of base for logarithms: Problem type 2}
% 106. type problem below this line

Consider the following equation.
$$\log_4{(13^{x+2})} = 3$$
Find the value of $x$.

Round your answer to the nearest thousandth.


%type problem above this line
\ifbool{answerKey}%
{
\vfill \textbf{Answer:}
%type answer below this line
$x = -0.379$



%type answer above this line
}%
{}
\vfill\hfill
{\scriptsize 8.0\theproblem}
\stepcounter{problem}
\end{minipage}}

%
%%%%%%%%%%%%%%%%%%%%%%%%%%%%%%%%%%%%%%
%

\fbox{\begin{minipage}[t][2.85 in]{.465 \textwidth}
\index{({ 8.0\theproblem} )\; Solving a multi-step equation involving a single logarithm}
% 107. type problem below this line

Solve for $x$.

$$\log_3{(-6-5x)} = 4$$




%type problem above this line
\ifbool{answerKey}%
{
\vfill \textbf{Answer:}
%type answer below this line

$$x = -\frac{87}{5}$$



%type answer above this line
}%
{}
\vfill\hfill
{\scriptsize 8.0\theproblem}
\stepcounter{problem}
\end{minipage}}%
%
%%%%%%%%%%%%%%%%%%%%%%%%%%%%%%%%%%%%%%
%
\fbox{\begin{minipage}[t][2.85 in]{.465 \textwidth}
\index{({ 8.0\theproblem} )\; Solving a multi-step equation involving natural logarithms}
% 108. type problem below this line

Solve for $x$.

Do not round any intermediate computations, and round your answer to the nearest hundredth.

$$5+\ln{(x+3)} = 4$$



%type problem above this line
\ifbool{answerKey}%
{
\vfill \textbf{Answer:}
%type answer below this line
$x = -2.63$



%type answer above this line
}%
{}
\vfill\hfill
{\scriptsize 8.0\theproblem}
\stepcounter{problem}
\end{minipage}}