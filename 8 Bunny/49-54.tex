%%%%%%%%%%%%%%%%%%%%%%%%%%%%%%%%%%%%%%
%
\fbox{\begin{minipage}[t][2.85 in]{.465 \textwidth}
\index{({ 8.0\theproblem} )\; Square root multiplication: Basic}
% 8.49. type problem below this line

Simplify.
$$\sqrt{2} \times \sqrt{50}$$



%type problem above this line
\ifbool{answerKey}%
{
\vfill \textbf{Answer:}
%type answer below this line
$10$



%type answer above this line
}%
{}
\vfill\hfill
{\scriptsize 8.0\theproblem}
\stepcounter{problem}
\end{minipage}}%
%
%%%%%%%%%%%%%%%%%%%%%%%%%%%%%%%%%%%%%%
%
\fbox{\begin{minipage}[t][2.85 in]{.465 \textwidth}
\index{({ 8.0\theproblem} )\; Introduction to simplifying a product of radical expressions: Univariate}
% 8.50. type problem below this line

Simplify. Assume that the variable represents a positive real number.

$$\sqrt{7x} \times \sqrt{10}$$



%type problem above this line
\ifbool{answerKey}%
{
\vfill \textbf{Answer:}
%type answer below this line
$\sqrt{70x}$



%type answer above this line
}%
{}
\vfill\hfill
{\scriptsize 8.0\theproblem}
\stepcounter{problem}
\end{minipage}}

%
%%%%%%%%%%%%%%%%%%%%%%%%%%%%%%%%%%%%%%
%

\fbox{\begin{minipage}[t][2.85 in]{.465 \textwidth}
\index{({ 8.0\theproblem} )\; Simplifying a product of radical expressions: Univariate}
% 8.51. type problem below this line

Simplify. Assume that the variable represents a positive real number.

$$\sqrt{3b} \sqrt{6b^8}$$



%type problem above this line
\ifbool{answerKey}%
{
\vfill \textbf{Answer:}
%type answer below this line

$$3b^4\sqrt{2b}$$



%type answer above this line
}%
{}
\vfill\hfill
{\scriptsize 8.0\theproblem}
\stepcounter{problem}
\end{minipage}}%
%
%%%%%%%%%%%%%%%%%%%%%%%%%%%%%%%%%%%%%%
%
\fbox{\begin{minipage}[t][2.85 in]{.465 \textwidth}
\index{({ 8.0\theproblem} )\; Introduction to simplifying a product involving square roots using the distributive property}
% 8.52. type problem below this line

Multiply. Simplify your answer as much as possible.

$$\sqrt{3}(12 + \sqrt{2})$$



%type problem above this line
\ifbool{answerKey}%
{
\vfill \textbf{Answer:}
%type answer below this line

$$12\sqrt{3} + \sqrt{6}$$



%type answer above this line
}%
{}
\vfill\hfill
{\scriptsize 8.0\theproblem}
\stepcounter{problem}
\end{minipage}}

%
%%%%%%%%%%%%%%%%%%%%%%%%%%%%%%%%%%%%%%
%

\fbox{\begin{minipage}[t][2.85 in]{.465 \textwidth}
\index{({ 8.0\theproblem} )\; Simplifying a product involving square roots using the distributive property: Basic}
% 8.53. type problem below this line

Multiply. Simplify your answer as much as possible.

$$\sqrt{3}(9\sqrt{15} + \sqrt{7})$$



%type problem above this line
\ifbool{answerKey}%
{
\vfill \textbf{Answer:}
%type answer below this line

$$27\sqrt{5} + \sqrt{21}$$



%type answer above this line
}%
{}
\vfill\hfill
{\scriptsize 8.0\theproblem}
\stepcounter{problem}
\end{minipage}}%
%
%%%%%%%%%%%%%%%%%%%%%%%%%%%%%%%%%%%%%%
%
\fbox{\begin{minipage}[t][2.85 in]{.465 \textwidth}
\index{({ 8.0\theproblem} )\; Classifying sums and products as rational or irrational}
% 8.54. type problem below this line

For each sum or product, determine whether the result is a rational number or an irrational number.
\vspace{.2 cm}

(a) $2 + \sqrt{21}$
\vspace{.2 cm}

(b) $\frac{10}{19} + \frac{5}{13}$
\vspace{.2 cm}

(c) $\sqrt{14} + 18$
\vspace{.2 cm}

(d) $15 \times \frac{9}{16}$




%type problem above this line
\ifbool{answerKey}%
{
\vfill \textbf{Answer:}
%type answer below this line

Rational: (b) and (d)

Irrational: (a) and (c)


%type answer above this line
}%
{}
\vfill\hfill
{\scriptsize 8.0\theproblem}
\stepcounter{problem}
\end{minipage}}