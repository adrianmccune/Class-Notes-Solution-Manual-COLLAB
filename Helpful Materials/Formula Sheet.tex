\addcontentsline{toc}{chapter}{Formulas}

\begin{center}
    {\huge Formulas}
\end{center}

\large Use this space to organize any formulas and other information you need to memorize for Math 100.

\vspace{.5 cm}

\begin{center}
\textbf{\large Geometry Formulas}
\vspace{.2 cm}
\end{center}

\large Use this space to record the geometry formulas (perimeter, area, volume, or surface area) you use in Math 100.
\vspace{.2 cm}

\begin{center}
    \begin{tabular}{| p{1.2 in} | p{1.4 in} | p{2.4 in} |}
        \hline
        \textbf{Shape} & \textbf{Formulas} & \textbf{Examples} \\
        \hline
        Circle & & \\
         & & \\
         & & \\
        \hline
        Rectangle & & \\
         & & \\
         & & \\
        \hline
        Triangle & & \\
         & & \\
         & & \\
        \hline
        Cylinder & & \\
         & & \\
         & & \\
        \hline
        Rectangular & & \\
        Prism & & \\
        & & \\
        \hline
         & & \\
         & & \\
         & & \\
        \hline
         & & \\
         & & \\
         & & \\
        \hline
    \end{tabular}

\newpage


\textbf{\large Linear Formulas}
\vspace{.2 cm}
\end{center}

\large Use this space to record the linear formulas (slope, equations for a line, vertical and horizontal lines, etc.) you use in Math 100.
\vspace{.2 cm}

\begin{center}
    \begin{tabular}{| p{1.6 in} | p{1.8 in} | p{1.8 in} |}
        \hline
        \textbf{Name} & \textbf{Formula} & \textbf{Examples} \\
        \hline
        Slope & & \\
        & & \\
         &  & \\
         &  & \\
        \hline
        Slope-Intercept &  & \\
        Form & & \\
         &  & \\
         &  & \\
        \hline
        Point-Slope &  & \\
        Form & & \\
        &  & \\
        & & \\
        \hline
        Standard &  & \\
        Form & & \\
        & & \\
        &  & \\
        \hline
        & & \\
        & & \\
        &  & \\
        & & \\
        \hline
    \end{tabular}


\end{center}

\newpage

\begin{center}
\textbf{\large Properties of Exponents}
\vspace{.2 cm}
\end{center}

\large Use this space to record the properties of exponents as you learn about them in Math 100.
\vspace{.2 cm}

\begin{center}

    \begin{tabular}{| p{1.8 in} | p{1.8 in} | p{1.8 in} |}
        \hline
        \textbf{Property} & \textbf{Definition} & \textbf{Example} \\
        \hline
        Product Rule & & \\
         & & \\
         & & \\
        \hline
        Quotient Rule & & \\
         & & \\
         & & \\
        \hline
        Power of a & & \\
        Power Rule & & \\
        & & \\
        \hline
        Power of a & & \\
        Product Rule & & \\
        & & \\
        \hline
        Power of a & & \\
        Fraction Rule & & \\
        & & \\
        \hline
        Zero Exponent & & \\
        Rule & & \\
        & & \\
        \hline
        Negative Exponent & & \\
        Rule (type 1) & & \\
        & & \\
        \hline
        Negative Exponent & & \\
        Rule (type 2) & & \\
        & & \\
        \hline
        & & \\
        & & \\
        & & \\
        \hline
    \end{tabular}

\newpage

\textbf{\large Metric Conversions}
\vspace{.2 cm}
\end{center}

\large Fill in the Metric Conversions chart below as you work to memorize these measurements. You can also use the rest of the space on this page to keep track of other formulas.
\vspace{.2 cm}

\begin{center}

\vspace{.5 cm}

\begin{tabular}{| p{.6 in} | p{.6 in} | p{.6 in} | p{.6 in} | p{.6 in} | p{.6 in} | p{.6 in} |}
    \hline
    & & & meter & & & \\
    milli- & centi- & deci- & gram & deca-  & hecta-  & kilo- \\
    & & & liter & & & \\
    \hline
    & & & & & & \\
    & & & 1 & & 100 & \\
    \hline
\end{tabular}

\end{center}
\newpage