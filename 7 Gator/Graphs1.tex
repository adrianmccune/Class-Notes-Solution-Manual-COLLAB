
\fbox{\begin{minipage}[t][4.25 in]{.465 \textwidth}
\index{({ 7.\theproblem} )\; 
%
Graphing a parabola of the form $y = (x-h)^2  + k$
%
}
% type problem below this line

Graph the parabola.

$$ y = (x+4)^2 + 2$$




%type problem above this line
\ifbool{answerKey}%
{
\vfill \textbf{Answer:}
%type answer below this line

\begin{center}
    \includegraphics[width=0.6\linewidth]{images/Gator G1.png}
\end{center}



%type answer above this line
}%
{

\begin{center}
    \includegraphics[width=0.7\linewidth]{images/Blank Graph.png}
\end{center}

}
\vfill
\hfill {\scriptsize 7.\theproblem}
\stepcounter{problem}
\end{minipage}}%
%
%%%%%%%%%%%%%%%%%%%%%%%%%%%%%%%%%%%%%%
%
\fbox{\begin{minipage}[t][4.25 in]{.465 \textwidth}
\index{({ 7.\theproblem} )\; 
%
Graphing an absolute value equation in the plane: Basic
%
}
% type problem below this line

Graph the equation.

$$ y = 3\vert x \vert + 2$$






%type problem above this line
\ifbool{answerKey}%
{
\vfill \textbf{Answer:}
%type answer below this line

\begin{center}
    \includegraphics[width=0.6\linewidth]{images/Gator G2.png}
\end{center}




%type answer above this line
}%
{

\begin{center}
    \includegraphics[width=0.7\linewidth]{images/Blank Graph.png}
\end{center}

}
\vfill
\hfill {\scriptsize 7.\theproblem}
\stepcounter{problem}
\end{minipage}}

%
%%%%%%%%%%%%%%%%%%%%%%%%%%%%%%%%%%%%%%
%

\fbox{\begin{minipage}[t][4.25 in]{.465 \textwidth}
\index{({ 7.\theproblem} )\; 
%
Writing an equation for a function after a vertical and horizontal translation
%
}
% type problem below this line

The graph of $f$ is translated a whole number of units horizontally and vertically to obtain the graph of $h$.

The function $f$ is defined by $f(x) = \sqrt{x}$.

Write down the expression for $h(x)$.

\begin{center}
    \includegraphics[width=0.6\linewidth]{images/Gator G3.png}
\end{center}




%type problem above this line
\ifbool{answerKey}%
{
\vfill \textbf{Answer:}
%type answer below this line

$h(x) = \sqrt{x + 3} + 2 $




%type answer above this line
}%
{}
\vfill
\hfill {\scriptsize 7.\theproblem}
\stepcounter{problem}
\end{minipage}}%
%
%%%%%%%%%%%%%%%%%%%%%%%%%%%%%%%%%%%%%%
%
\fbox{\begin{minipage}[t][4.25 in]{.465 \textwidth}
\index{({ 7.\theproblem} )\; 
%
Finding where a function is increasing, decreasing, or constant given the graph: Interval notation
%
}
% type problem below this line

Determine the interval(s) on which the function is (strictly) increasing.

Write your answer as an interval or list of intervals.
When writing a list of intervals, make sure to separate each interval with a comma and to use as few intervals as possible. Write ``None" if applicable.


\begin{center}
    \includegraphics[width=0.5\linewidth]{images/Gator G4.png}
\end{center}



%type problem above this line
\ifbool{answerKey}%
{
\vfill \textbf{Answer:}
%type answer below this line

$$ (-3, 1), (3, 5) $$




%type answer above this line
}%
{}
\vfill
\hfill {\scriptsize 7.\theproblem}
\stepcounter{problem}
\end{minipage}}