%%%%%%%%%%%%%%%%%%%%%%%%%%%%%%%%%%%%%%
%
\fbox{\begin{minipage}[t][2.85 in]{.465 \textwidth}
\index{({ 7.\theproblem} )\; Introduction to simplifying a radical expression with an odd exponent}
% type problem below this line
Simplify.
$$\sqrt{w^7}$$
Assume that the variable represents a positive real number.




%type problem above this line
\ifbool{answerKey}%
{
\vfill \textbf{Answer:}
%type answer below this line
$w^3\sqrt{w}$




%type answer above this line
}%
{}
\vfill\hfill
{\scriptsize 7.\theproblem}
\stepcounter{problem}
\end{minipage}}%
%
%%%%%%%%%%%%%%%%%%%%%%%%%%%%%%%%%%%%%%
%
\fbox{\begin{minipage}[t][2.85 in]{.465 \textwidth}
\index{({ 7.\theproblem} )\; Simplifying a radical expression with an odd exponent}
% type problem below this line
Simplify.
$$\sqrt{48x^{15}}$$
Assume that the variable represents a positive real number.




%type problem above this line
\ifbool{answerKey}%
{
\vfill \textbf{Answer:}
%type answer below this line
$4x^7\sqrt{3x}$




%type answer above this line
}%
{}
\vfill\hfill
{\scriptsize 7.\theproblem}
\stepcounter{problem}
\end{minipage}}

%
%%%%%%%%%%%%%%%%%%%%%%%%%%%%%%%%%%%%%%
%

\fbox{\begin{minipage}[t][2.85 in]{.465 \textwidth}
\index{({ 7.\theproblem} )\; Solving an equation of the form $x^2 = a$ using the square root property
}
% type problem below this line
Solve $u^2=-16$, where $u$ is a real number.

Simplify your answer as much as possible.

If there is more than one solution, separate them with commas.

If there is no solution, write ``No solution".





%type problem above this line
\ifbool{answerKey}%
{
\vfill \textbf{Answer:}
%type answer below this line
No solution




%type answer above this line
}%
{}
\vfill\hfill
{\scriptsize 7.\theproblem}
\stepcounter{problem}
\end{minipage}}%
%
%%%%%%%%%%%%%%%%%%%%%%%%%%%%%%%%%%%%%%
%
\fbox{\begin{minipage}[t][2.85 in]{.465 \textwidth}
\index{({ 7.\theproblem} )\; Solving a quadratic equation using the square root property: Exact answers, basic}
% type problem below this line
Solve $x^2=54$, where $x$ is a real number.

Simplify your answer as much as possible.

If there is more than one solution, separate them with commas.

If there is no solution, write ``No solution."




%type problem above this line
\ifbool{answerKey}%
{
\vfill \textbf{Answer:}
%type answer below this line
$x = 3\sqrt{6}, -3\sqrt{6}$



%type answer above this line
}%
{}
\vfill\hfill
{\scriptsize 7.\theproblem}
\stepcounter{problem}
\end{minipage}}

%
%%%%%%%%%%%%%%%%%%%%%%%%%%%%%%%%%%%%%%
%

\fbox{\begin{minipage}[t][2.85 in]{.465 \textwidth}
\index{({ 7.\theproblem} )\; Applying the quadratic formula: Exact answers}
% type problem below this line
Use the quadratic formula to solve for $x$.
$$3x^2-9x+4=0$$
(If there is more than one solution, separate them with commas.)






%type problem above this line
\ifbool{answerKey}%
{
\vfill \textbf{Answer:}
%type answer below this line
$x=\frac{9+\sqrt{33}}{6}, \frac{9-\sqrt{33}}{6}$




%type answer above this line
}%
{}
\vfill\hfill
{\scriptsize 7.\theproblem}
\stepcounter{problem}
\end{minipage}}%
%
%%%%%%%%%%%%%%%%%%%%%%%%%%%%%%%%%%%%%%
%
\fbox{\begin{minipage}[t][2.85 in]{.465 \textwidth}
\index{({ 7.\theproblem} )\; Discriminant of a quadratic equation}
% type problem below this line
Compute the value of the discriminant and give the number of real solutions of the quadratic equation.
$$ -2x^2 + 3x + 6 = 0 $$

\vspace{0.25cm}

Discriminant: 
\vspace{0.5cm}

Number of Real Solutions: 




%type problem above this line
\ifbool{answerKey}%
{
\vfill \textbf{Answer:}
%type answer below this line

Discriminant: 57

Number of Real Solutions: 2



%type answer above this line
}%
{}
\vfill\hfill
{\scriptsize 7.\theproblem}
\stepcounter{problem}
\end{minipage}}