%%%%%%%%%%%%%%%%%%%%%%%%%%%%%%%%%%%%%%
%
\fbox{\begin{minipage}[t][2.85 in]{.465 \textwidth}
\index{({ 3.0\theproblem} )\; Least common multiple of 3 numbers}
% type problem below this line

Find the least common multiple of $3, 5$, and $2$.




%type problem above this line
\ifbool{answerKey}%
{
\vfill \textbf{Answer:}
%type answer below this line
$30$



%type answer above this line
}%
{}
\vfill\hfill
{\scriptsize 3.0\theproblem}
\stepcounter{problem}
\end{minipage}}%
%
%%%%%%%%%%%%%%%%%%%%%%%%%%%%%%%%%%%%%%
%
\fbox{\begin{minipage}[t][2.85 in]{.465 \textwidth}
\index{({ 3.0\theproblem} )\; Finding the next terms of a geometric sequence with whole numbers}
% type problem below this line
The first three terms of a geometric sequence are as follows.
$$81,\; 27,\; 9$$
Find the next two terms of this sequence.
 






%type problem above this line
\ifbool{answerKey}%
{
\vfill \textbf{Answer:}
%type answer below this line
$3,\; 1$



%type answer above this line
}%
{}
\vfill\hfill
{\scriptsize 3.0\theproblem}
\stepcounter{problem}
\end{minipage}}

%
%%%%%%%%%%%%%%%%%%%%%%%%%%%%%%%%%%%%%%
%

\fbox{\begin{minipage}[t][2.85 in]{.465 \textwidth}
\index{({ 3.0\theproblem} )\; Converting a decimal to a mixed number and an improper fraction in simplest form: Basic}
% type problem below this line
Write $2.25$ as a mixed number and as an improper fraction.
Write your answers in simplest form.
\vspace{.2cm}

mixed number:
\vspace{.2cm}

improper fraction: 





%type problem above this line
\ifbool{answerKey}%
{
\vfill \textbf{Answer:}
%type answer below this line

mixed number: $2 \frac{1}{4}$
\vspace{.2cm}

improper fraction: $\frac{9}{4}$


%type answer above this line
}%
{}
\vfill\hfill
{\scriptsize 3.0\theproblem}
\stepcounter{problem}
\end{minipage}}%
%
%%%%%%%%%%%%%%%%%%%%%%%%%%%%%%%%%%%%%%
%
\fbox{\begin{minipage}[t][2.85 in]{.465 \textwidth}
\index{({ 3.0\theproblem} )\; Converting a decimal to a mixed number and an improper fraction in simplest form: Advanced}
% type problem below this line
Write $5.075$ as a mixed number and as an improper fraction.
Write your answers in simplest form.
\vspace{.2cm}

mixed number:
\vspace{.2cm}

improper fraction: 





%type problem above this line
\ifbool{answerKey}%
{
\vfill \textbf{Answer:}
%type answer below this line

mixed number: $5 \frac{3}{40}$
\vspace{.2cm}

improper fraction: $\frac{203}{40}$



%type answer above this line
}%
{}
\vfill\hfill
{\scriptsize 3.0\theproblem}
\stepcounter{problem}
\end{minipage}}

%
%%%%%%%%%%%%%%%%%%%%%%%%%%%%%%%%%%%%%%
%

\fbox{\begin{minipage}[t][2.85 in]{.465 \textwidth}
\index{({ 3.0\theproblem} )\; Using a calculator to convert a fraction to a rounded decimal}
% type problem below this line
Use a calculator to write $\frac{13}{14}$ as a decimal rounded to the nearest tenth.





%type problem above this line
\ifbool{answerKey}%
{
\vfill \textbf{Answer:}
%type answer below this line
$0.9$



%type answer above this line
}%
{}
\vfill\hfill
{\scriptsize 3.0\theproblem}
\stepcounter{problem}
\end{minipage}}%
%
%%%%%%%%%%%%%%%%%%%%%%%%%%%%%%%%%%%%%%
%
\fbox{\begin{minipage}[t][2.85 in]{.465 \textwidth}
\index{({ 3.0\theproblem} )\; Solving a proportion of the form x/a = b/c}
% type problem below this line
Solve the following proportion for $x$.
$$ \frac{x}{12} = \frac{7}{17}$$
Round your answer to the nearest tenth.
 






%type problem above this line
\ifbool{answerKey}%
{
\vfill \textbf{Answer:}
%type answer below this line
$x = 4.9$



%type answer above this line
}%
{}
\vfill\hfill
{\scriptsize 3.0\theproblem}
\stepcounter{problem}
\end{minipage}}