%%%%%%%%%%%%%%%%%%%%%%%%%%%%%%%%%%%%%%
\fbox{%
  \begin{minipage}[t]{.465\textwidth}
    \index{({ 1.0\theproblem})\; Rounding to thousands, ten thousands, or hundred thousands}

    % Problem
    Round $281{,}380$ to the nearest ten thousand.

    \ifbool{answerKey}{
      \textbf{Answer:} $280{,}000$

      \textbf{Explanation:}
      \begin{itemize}
          \item Identify the digit in the ten-thousands place: 8.
          \item Look at the digit to the right (thousands place): 1.
          \item Since 1 < 5, round down.
          \item So $281{,}380$ rounded to the nearest ten thousand is $280{,}000$.
      \end{itemize}
    }{}

    \vfill\hfill{\scriptsize 1.0\theproblem}
    \stepcounter{problem}
  \end{minipage}%
}
%%%%%%%%%%%%%%%%%%%%%%%%%%%%%%%%%%%%%%
\fbox{%
  \begin{minipage}[t]{.465\textwidth}
    \index{({ 1.0\theproblem})\; Estimating a sum of whole numbers: Problem type 2}

    % Problem
    Estimate $9082 + 2542$ by first rounding each number to the nearest thousand.

    \ifbool{answerKey}{
      \textbf{Answer:} $12{,}000$

      \textbf{Explanation:}
      \begin{itemize}
          \item Round $9082$ to the nearest thousand: $9000$.
          \item Round $2542$ to the nearest thousand: $3000$.
          \item Add the rounded numbers: $9000 + 3000 = 12{,}000$.
      \end{itemize}
    }{}

    \vfill\hfill{\scriptsize 1.0\theproblem}
    \stepcounter{problem}
  \end{minipage}%
}
%%%%%%%%%%%%%%%%%%%%%%%%%%%%%%%%%%%%%%
\fbox{%
  \begin{minipage}[t]{.465\textwidth}
    \index{({ 1.0\theproblem})\; Estimating a difference of whole numbers: Problem type 2}

    % Problem
    Estimate $81{,}146 - 22{,}010$ by first rounding each number to the nearest thousand.

    \ifbool{answerKey}{
      \textbf{Answer:} $59{,}000$

      \textbf{Explanation:}
      \begin{itemize}
          \item Round $81{,}146$ to the nearest thousand: $81{,}000$.
          \item Round $22{,}010$ to the nearest thousand: $22{,}000$.
          \item Subtract the rounded numbers: $81{,}000 - 22{,}000 = 59{,}000$.
      \end{itemize}
    }{}

    \vfill\hfill{\scriptsize 1.0\theproblem}
    \stepcounter{problem}
  \end{minipage}%
}
%%%%%%%%%%%%%%%%%%%%%%%%%%%%%%%%%%%%%%
\fbox{%
  \begin{minipage}[t]{.465\textwidth}
    \index{({ 1.0\theproblem})\; Introduction to fractions}

    % Problem
    The circle below is cut into $12$ equal slices.
    What fraction of the circle is shaded?
    \begin{center}
        \includegraphics[width=0.4\linewidth]{images/hippo 52.png}
    \end{center}

    \ifbool{answerKey}{
      \textbf{Answer:} $\frac{5}{12}$

      \textbf{Explanation:}
      \begin{itemize}
          \item Count the shaded slices: 5.
          \item Total slices: 12.
          \item Write as a fraction: $\frac{5}{12}$.
      \end{itemize}
    }{}

    \vfill\hfill{\scriptsize 1.0\theproblem}
    \stepcounter{problem}
  \end{minipage}%
}
%%%%%%%%%%%%%%%%%%%%%%%%%%%%%%%%%%%%%%
\fbox{%
  \begin{minipage}[t]{.465\textwidth}
    \index{({ 1.0\theproblem})\; Understanding equivalent fractions}

    % Problem
    The pie below is cut into $8$ equal slices.
    Shade $1/4$ of this pie.
    \begin{center}
        \includegraphics[width=0.33\linewidth]{images/hippo 53.png}
    \end{center}

    \ifbool{answerKey}{
      \textbf{Answer:}
      \begin{center}
          \includegraphics[width=0.15\linewidth]{images/hippo 53 ans.png}
      \end{center}

      \textbf{Explanation:}
      \begin{itemize}
          \item $1/4$ means one part out of four equal parts.
          \item The pie has 8 slices, so divide into groups of 2 slices each.
          \item Shade 2 slices to represent $1/4$.
      \end{itemize}
    }{}

    \vfill\hfill{\scriptsize 1.0\theproblem}
    \stepcounter{problem}
  \end{minipage}%
}
%%%%%%%%%%%%%%%%%%%%%%%%%%%%%%%%%%%%%%
\fbox{%
  \begin{minipage}[t]{.465\textwidth}
    \index{({ 1.0\theproblem})\; Equivalent fractions}

    % Problem
    Fill in the blank to make the two fractions equivalent.

    $$
    \frac{}{\,30\,} = \frac{\,\,3\,\,}{5}
    $$

    \ifbool{answerKey}{
      \textbf{Answer:} $18$

      \textbf{Explanation:}
      \begin{itemize}
          \item Compare denominators: 30 and 5.
          \item Multiply 5 by 6 to get 30.
          \item Multiply numerator by the same factor: $3 \times 6 = 18$.
          \item So the missing numerator is 18.
      \end{itemize}
    }{}

    \vfill\hfill{\scriptsize 1.0\theproblem}
    \stepcounter{problem}
  \end{minipage}%
}
%%%%%%%%%%%%%%%%%%%%%%%%%%%%%%%%%%%%%%