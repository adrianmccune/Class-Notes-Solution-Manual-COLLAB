%%%%%%%%%%%%%%%%%%%%%%%%%%%%%%%%%%%%%%
\fbox{%
  \begin{minipage}[t]{.465\textwidth}
    \index{({ 1.0\theproblem})\; Decimal place value: Tenths and hundredths}

    % Problem
    Give the digits in the tenths place and the hundredths place.
    $$ 64.01 $$

    tenths:

    hundredths:

    \ifbool{answerKey}{
      \textbf{Answer:}
      tenths: $0$ \\
      hundredths: $1$

      \textbf{Explanation:}
      \begin{itemize}
          \item The tenths place is the first digit to the right of the decimal point.
          \item In $64.01$, the digit in the tenths place is $0$.
          \item The hundredths place is the second digit to the right of the decimal point.
          \item In $64.01$, the digit in the hundredths place is $1$.
      \end{itemize}
    }{}

    \vfill\hfill{\scriptsize 1.0\theproblem}
    \stepcounter{problem}
  \end{minipage}%
}
%%%%%%%%%%%%%%%%%%%%%%%%%%%%%%%%%%%%%%
\fbox{%
  \begin{minipage}[t]{.465\textwidth}
    \index{({ 1.0\theproblem})\; Writing a decimal number less than 1 given its name}

    % Problem
    Write the following number in standard decimal form.
    \begin{center}
        \textit{two hundredths}
    \end{center}

    \ifbool{answerKey}{
      \textbf{Answer:} $0.02$

      \textbf{Explanation:}
      \begin{itemize}
          \item "Two hundredths" means 2 parts out of 100.
          \item The hundredths place is the second digit after the decimal point.
          \item So, the decimal form is $0.02$.
      \end{itemize}
    }{}

    \vfill\hfill{\scriptsize 1.0\theproblem}
    \stepcounter{problem}
  \end{minipage}%
}
%%%%%%%%%%%%%%%%%%%%%%%%%%%%%%%%%%%%%%
\fbox{%
  \begin{minipage}[t]{.465\textwidth}
    \index{({ 1.0\theproblem})\; Writing a decimal number greater than 1 given its name}

    % Problem
    Write the following number in standard decimal form.
    \begin{center}
        \textit{eleven and five hundredths}
    \end{center}

    \ifbool{answerKey}{
      \textbf{Answer:} $11.05$

      \textbf{Explanation:}
      \begin{itemize}
          \item "Eleven" is the whole number part.
          \item "Five hundredths" means 5 parts out of 100.
          \item Place 5 in the hundredths position: $11.05$.
      \end{itemize}
    }{}

    \vfill\hfill{\scriptsize 1.0\theproblem}
    \stepcounter{problem}
  \end{minipage}%
}
%%%%%%%%%%%%%%%%%%%%%%%%%%%%%%%%%%%%%%
\fbox{%
  \begin{minipage}[t]{.465\textwidth}
    \index{({ 1.0\theproblem})\; Reading decimal position on a number line: Tenths}

    % Problem
    What is the location of \textit{C} on the decimal number line below?
    Write your answer as a decimal.
    \begin{center}
        \includegraphics[width=1.0\linewidth]{images/hippo 76.png}
    \end{center}

    \ifbool{answerKey}{
      \textbf{Answer:} $2.4$

      \textbf{Explanation:}
      \begin{itemize}
          \item The number line is divided into tenths (0.1 increments).
          \item Find point C between 2.3 and 2.5.
          \item C is at 2.4.
      \end{itemize}
    }{}

    \vfill\hfill{\scriptsize 1.0\theproblem}
    \stepcounter{problem}
  \end{minipage}%
}
%%%%%%%%%%%%%%%%%%%%%%%%%%%%%%%%%%%%%%
\fbox{%
  \begin{minipage}[t]{.465\textwidth}
    \index{({ 1.0\theproblem})\; Reading decimal position on a number line: Hundredths}

    % Problem
    What is the location of \textit{C} on the decimal number line below?
    Write your answer as a decimal.
    \begin{center}
        \includegraphics[width=1.0\linewidth]{images/hippo 77.png}
    \end{center}

    \ifbool{answerKey}{
      \textbf{Answer:} $2.52$

      \textbf{Explanation:}
      \begin{itemize}
          \item The number line is divided into hundredths (0.01 increments).
          \item Find point C between 2.51 and 2.53.
          \item C is at 2.52.
      \end{itemize}
    }{}

    \vfill\hfill{\scriptsize 1.0\theproblem}
    \stepcounter{problem}
  \end{minipage}%
}
%%%%%%%%%%%%%%%%%%%%%%%%%%%%%%%%%%%%%%
\fbox{%
  \begin{minipage}[t]{.465\textwidth}
    \index{({ 1.0\theproblem})\; Choosing U.S. Customary measurement units}

    % Problem
    Fill in the blank with the correct units.

    (a) Chau drank about $2\;\underline{\qquad}$ of juice with lunch. (cups/gallons)

    (b) A whale weighs about $45\;\underline{\qquad}$. (ounces/pounds/tons)

    (c) Kira's family went on a hike that was about $5\;\underline{\qquad}$. (inches/feet/yards/miles)

    \ifbool{answerKey}{
      \textbf{Answer:}
      a) cups \\
      b) tons \\
      c) miles

      \textbf{Explanation:}
      \begin{itemize}
          \item (a) A person typically drinks a few cups of juice, not gallons.
          \item (b) Whales are extremely heavy, so their weight is measured in tons.
          \item (c) A hike is several miles long, not inches, feet, or yards.
      \end{itemize}
    }{}

    \vfill\hfill{\scriptsize 1.0\theproblem}
    \stepcounter{problem}
  \end{minipage}%
}
%%%%%%%%%%%%%%%%%%%%%%%%%%%%%%%%%%%%%%