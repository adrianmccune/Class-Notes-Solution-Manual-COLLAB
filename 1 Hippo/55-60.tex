%%%%%%%%%%%%%%%%%%%%%%%%%%%%%%%%%%%%%%
\fbox{%
  \begin{minipage}[t]{.465\textwidth}
    \index{({ 1.0\theproblem})\; Introduction to simplifying a fraction}

    % Problem
    Write $\frac{10}{15}$ in simplest form.

    \ifbool{answerKey}{
      \textbf{Answer:} $\frac{2}{3}$

      \textbf{Explanation:}
      \begin{itemize}
          \item Find the greatest common divisor (GCD) of 10 and 15, which is 5.
          \item Divide numerator and denominator by 5: $10 \div 5 = 2$, $15 \div 5 = 3$.
          \item So the fraction in simplest form is $\frac{2}{3}$.
      \end{itemize}
    }{}

    \hfill{\scriptsize 1.0\theproblem}
    \stepcounter{problem}
  \end{minipage}%
}
%%%%%%%%%%%%%%%%%%%%%%%%%%%%%%%%%%%%%%
\fbox{%
  \begin{minipage}[t]{.465\textwidth}
    \index{({ 1.0\theproblem})\; Simplifying a fraction}

    % Problem
    Write the fraction $\frac{5}{20}$ in simplest form.

    \ifbool{answerKey}{
      \textbf{Answer:} $\frac{1}{4}$

      \textbf{Explanation:}
      \begin{itemize}
          \item Find the GCD of 5 and 20, which is 5.
          \item Divide numerator and denominator by 5: $5 \div 5 = 1$, $20 \div 5 = 4$.
          \item So the fraction in simplest form is $\frac{1}{4}$.
      \end{itemize}
    }{}

    \hfill{\scriptsize 1.0\theproblem}
    \stepcounter{problem}
  \end{minipage}%
}
%%%%%%%%%%%%%%%%%%%%%%%%%%%%%%%%%%%%%%
\fbox{%
  \begin{minipage}[t]{.465\textwidth}
    \index{({ 1.0\theproblem})\; Fractional position on a number line}

    % Problem
    What is the position of E on the number line below?
    Write your answer as a fraction or mixed number.
    \begin{center}
        \includegraphics[width=1.0\linewidth]{images/hippo 57.png}
    \end{center}

    \ifbool{answerKey}{
      \textbf{Answer:} $1\frac{2}{4}$

      \textbf{Explanation:}
      \begin{itemize}
          \item Identify the whole number part: E is past 1.
          \item Count the fractional parts: E is at 2 out of 4 equal sections beyond 1.
          \item So the position is $1\frac{2}{4}$ (which could also simplify to $1\frac{1}{2}$).
      \end{itemize}
    }{}

    \hfill{\scriptsize 1.0\theproblem}
    \stepcounter{problem}
  \end{minipage}%
}
%%%%%%%%%%%%%%%%%%%%%%%%%%%%%%%%%%%%%%
\fbox{%
  \begin{minipage}[t]{.465\textwidth}
    \index{({ 1.0\theproblem})\; Plotting fractions on a number line}

    % Problem
    \begin{center}
        \includegraphics[width=1.0\linewidth]{images/hippo 58.png}
    \end{center}

    \ifbool{answerKey}{
      \textbf{Answer:}
      \begin{center}
          \includegraphics[width=1.0\linewidth]{images/hippo 58 ans.png}
      \end{center}

      \textbf{Explanation:}
      \begin{itemize}
          \item Divide the number line into equal parts based on the denominators.
          \item Place each fraction at its correct position according to its value.
          \item Verify that fractions with the same denominator are evenly spaced.
      \end{itemize}
    }{}

    \hfill{\scriptsize 1.0\theproblem}
    \stepcounter{problem}
  \end{minipage}%
}
%%%%%%%%%%%%%%%%%%%%%%%%%%%%%%%%%%%%%%
\fbox{%
  \begin{minipage}[t]{.465\textwidth}
    \index{({ 1.0\theproblem})\; Ordering fractions with the same denominator}

    % Problem
    Order these fractions from least to greatest.

    $$
    \frac{2}{11} \qquad \frac{9}{11} \qquad \frac{7}{11}
    $$

    \vspace{.5 cm}

    $$
    \underline{\quad\qquad} < \underline{\quad\qquad} < \underline{\quad\qquad}
    $$

    \ifbool{answerKey}{
      \textbf{Answer:} $\frac{2}{11} < \frac{7}{11} < \frac{9}{11}$

      \textbf{Explanation:}
      \begin{itemize}
          \item All fractions have the same denominator (11), so compare numerators.
          \item Order numerators: 2, 7, 9.
          \item So the order is $\frac{2}{11} < \frac{7}{11} < \frac{9}{11}$.
      \end{itemize}
    }{}

    \hfill{\scriptsize 1.0\theproblem}
    \stepcounter{problem}
  \end{minipage}%
}
%%%%%%%%%%%%%%%%%%%%%%%%%%%%%%%%%%%%%%
\fbox{%
  \begin{minipage}[t]{.465\textwidth}
    \index{({ 1.0\theproblem})\; Ordering fractions with the same numerator}

    % Problem
    Order these fractions from least to greatest.

    $$
    \frac{2}{9} \qquad \frac{2}{3} \qquad \frac{2}{6}
    $$

    \vspace{.5 cm}

    $$
    \underline{\quad\qquad} < \underline{\quad\qquad} < \underline{\quad\qquad}
    $$

    \ifbool{answerKey}{
      \textbf{Answer:} $\frac{2}{9} < \frac{2}{6} < \frac{2}{3}$

      \textbf{Explanation:}
      \begin{itemize}
          \item All fractions have the same numerator (2), so compare denominators.
          \item Larger denominator means smaller fraction.
          \item Order denominators: 9, 6, 3.
          \item So the order is $\frac{2}{9} < \frac{2}{6} < \frac{2}{3}$.
      \end{itemize}
    }{}

    \hfill{\scriptsize 1.0\theproblem}
    \stepcounter{problem}
  \end{minipage}%
}
%%%%%%%%%%%%%%%%%%%%%%%%%%%%%%%%%%%%%%