%%%%%%%%%%%%%%%%%%%%%%%%%%%%%%%%%%%%%%
%
\fbox{%
  \begin{minipage}[t]{.465\textwidth}
    \index{({ 1.0\theproblem})\; Product of a unit fraction and a whole number}
    % Problem
    What is $\frac{1}{4}$ of $20$?

    \ifbool{answerKey}{
      \textbf{Answer:}
      $5$

      \textbf{Explanation:}
      \begin{itemize}
          \item To find $\frac{1}{4}$ of 20, multiply 20 by $\frac{1}{4}$.
          \item $20 \times \frac{1}{4} = \frac{20}{4}$.
          \item Simplify: $\frac{20}{4} = 5$.
      \end{itemize}
    }{}

    \vfill\hfill{\scriptsize 1.0\theproblem}
    \stepcounter{problem}
  \end{minipage}%
}
%%%%%%%%%%%%%%%%%%%%%%%%%%%%%%%%%%%%%%
%
\fbox{%
  \begin{minipage}[t]{.465\textwidth}
    \index{({ 1.0\theproblem})\; Product of a fraction and a whole number: Problem type 1}
    % Problem
    Multiply.
    $$ \frac{2}{5} \times 45 $$

    \ifbool{answerKey}{
      \textbf{Answer:}
      $18$

      \textbf{Explanation:}
      \begin{itemize}
          \item Multiply the whole number by the numerator: $45 \times 2 = 90$.
          \item Divide by the denominator: $90 \div 5 = 18$.
          \item So the product is 18.
      \end{itemize}
    }{}

    \vfill\hfill{\scriptsize 1.0\theproblem}
    \stepcounter{problem}
  \end{minipage}%
}
%%%%%%%%%%%%%%%%%%%%%%%%%%%%%%%%%%%%%%
%
\fbox{%
  \begin{minipage}[t]{.465\textwidth}
    \index{({ 1.0\theproblem})\; Product of a fraction and a whole number: Problem type 2}
    % Problem
    Multiply. Write your answer as a fraction in simplest form.
    $$ \frac{3}{26} \times 8 $$

    \ifbool{answerKey}{
      \textbf{Answer:}
      $\frac{12}{13}$

      \textbf{Explanation:}
      \begin{itemize}
          \item Multiply the numerator by the whole number: $3 \times 8 = 24$.
          \item Keep the denominator: $26$.
          \item Fraction is $\frac{24}{26}$.
          \item Simplify by dividing numerator and denominator by 2: $\frac{24}{26} = \frac{12}{13}$.
      \end{itemize}
    }{}

    \vfill\hfill{\scriptsize 1.0\theproblem}
    \stepcounter{problem}
  \end{minipage}%
}
%%%%%%%%%%%%%%%%%%%%%%%%%%%%%%%%%%%%%%
%
\fbox{%
  \begin{minipage}[t]{.465\textwidth}
    \index{({ 1.0\theproblem})\; Addition or subtraction of fractions with the same denominator}
    % Problem
    Subtract.
    $$ \frac{5}{11} - \frac{2}{11} $$

    \ifbool{answerKey}{
      \textbf{Answer:}
      $\frac{3}{11}$

      \textbf{Explanation:}
      \begin{itemize}
          \item Fractions have the same denominator, so subtract numerators.
          \item $5 - 2 = 3$.
          \item Keep the denominator: $11$.
          \item So the result is $\frac{3}{11}$.
      \end{itemize}
    }{}

    \vfill\hfill{\scriptsize 1.0\theproblem}
    \stepcounter{problem}
  \end{minipage}%
}
%%%%%%%%%%%%%%%%%%%%%%%%%%%%%%%%%%%%%%
%
\fbox{%
  \begin{minipage}[t]{.465\textwidth}
    \index{({ 1.0\theproblem})\; Addition or subtraction of fractions with the same denominator and simplification}
    % Problem
    Add. Write your answer as a fraction in simplest form.
    $$ \frac{3}{8} + \frac{1}{8} $$

    \ifbool{answerKey}{
      \textbf{Answer:}
      $\frac{1}{2}$

      \textbf{Explanation:}
      \begin{itemize}
          \item Fractions have the same denominator, so add numerators.
          \item $3 + 1 = 4$.
          \item Keep the denominator: $8$.
          \item Fraction is $\frac{4}{8}$.
          \item Simplify: $\frac{4}{8} = \frac{1}{2}$.
      \end{itemize}
    }{}

    \vfill\hfill{\scriptsize 1.0\theproblem}
    \stepcounter{problem}
  \end{minipage}%
}
%%%%%%%%%%%%%%%%%%%%%%%%%%%%%%%%%%%%%%
%
\fbox{%
  \begin{minipage}[t]{.465\textwidth}
    \index{({ 1.0\theproblem})\; Writing a mixed number and an improper fraction for a shaded region}
    % Problem
    Each circle counts as one whole.

    Write a mixed number giving the amount shaded.
    Then, write this amount as an improper fraction.
    \begin{center}
        \includegraphics[width=0.6\linewidth]{images/hippo 66.png}
    \end{center}
    a) Mixed number:

    b) Improper fraction:

    \ifbool{answerKey}{
      \textbf{Answer:}
      a) Mixed number: $3 \frac{1}{5}$

      b) Improper fraction: $\frac{16}{5}$

      \textbf{Explanation:}
      \begin{itemize}
          \item Count the whole circles: 3.
          \item Count the shaded part of the last circle: $\frac{1}{5}$.
          \item Mixed number: $3 \frac{1}{5}$.
          \item Convert to improper fraction: $(3 \times 5) + 1 = 15 + 1 = 16$.
          \item So improper fraction is $\frac{16}{5}$.
      \end{itemize}
    }{}

    \vfill\hfill{\scriptsize 1.0\theproblem}
    \stepcounter{problem}
  \end{minipage}%
}
%%%%%%%%%%%%%%%%%%%%%%%%%%%%%%%%%%%%%%